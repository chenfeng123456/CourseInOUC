\documentclass[UTF8,a4paper]{ctexart}

%页面边距
\usepackage{geometry}
\geometry{a4paper,left=2cm,right=2cm,top=1.5cm,bottom=1.5cm}

%底部对齐
%\usepackage{balance}
%\balance
\usepackage{flushend}

%在multicols环境中使用figure环境,使用[H]参数
\usepackage{float}

\usepackage[justification=centering]{caption} 

%设置页码
\pagenumbering{arabic}
\pagestyle{plain}

%分栏
\usepackage{multicol}

%引入了一些改进的数学环境,如align
\usepackage{amsmath}

%得到引用的标题内容
\usepackage{nameref} 

%照片
\usepackage{graphicx}

%添加首行缩进,两个字符
\usepackage{indentfirst}
\setlength{\parindent}{2em}

%多行公式一个编号
\usepackage{amsmath}

%文献引用,标准类型为plain
%\usepackage[hyperref=true,backend=biber,sorting=none,backref=true]{biblatex}
%\addbibresource{ref.bib}
\bibliographystyle{plain}
\usepackage{cite}

%超链接
\usepackage[linkcolor=yellow,citecolor=red,backref=page]{hyperref}
\hypersetup{
bookmarks=true,
colorlinks=true,
linkcolor=blue
}

\title{传统民歌如何才能更好地融入当代青年人的生活}

\author{姓名:鲁国锐 \protect\newline
\and 学号:17020021031 \\
\and 专业:电子信息科学与技术}





\begin{document}
\maketitle

\begin{multicols}{2}

	\begin{abstract}
		\indent \textit{中华传统民歌有着丰厚的底蕴,历经时间沉淀洗礼,是我国传统文化中不可忽视的精粹。然而在新的时代背景与文化环境下,我国的传统民歌正受到一定冲击。相较于流行音乐而言,传统民歌的发展似乎并不繁盛。因此,如何让传统民歌融入到人们的生活当中,尤其是青年人的生活当中,就显得十分关键了。在此背景下,本文首先给出了传统民歌的定义,简要分析了何为传统民歌;然后为传统民歌的发展现状做出一个简明的描述;最后结合传统民歌的特点与发展现状给出如何让民歌融入当代青年人生活的建议。}
	\end{abstract}
	
	
	\section{何为传统民歌}\label{定义}
		\indent 关于传统民歌的定义,有着不同的说法,这里截取一段流传较为广泛地定义:“传统民歌,是指每个民族的传统歌曲,每个民族的先民都有他们自古代已有的歌曲,这些歌绝大部分都不知道谁是作者,而以口头传播,一传十十传百,一代传一代的传下去至今。”这是一个大致的定义,并不能涵盖所有的情况。对于传统民歌,下一个精确的定义似乎并不容易。但定义只是辅助人们了解认识一个新事物的手段,不能完全依赖定义来进行对事物的认知。特别地,对于民歌而言,“若单凭一两个定义看民歌会限制了对民歌全面的认识, 其发展演变的过程及和其他乐种的关系”\cite{yu1994给中国民歌下个定义}。所以,本文也不在这个问题上做过多讨论,仅给出上述的大致定义,以供了解。
	
	
	
	
	\section{传统民歌的发展现状}\label{现状}
		\indent 民歌可以说是中国产生最早的一种声乐题材\cite{yang2017浅析当代民歌艺术},其发展与传承受我国各地方不同的不同的文化底蕴影响,有着非常丰富的种类与流派。但不管是哪种类型,都有着自身独特的魅力。民歌当中蕴含着浓厚的乡土文化和地方特色, 歌声淳朴自然, 能够表达出歌唱者内心最为真挚的情感\cite{ge2017中国民歌唱法的现状};这也得益于民歌基于地方特征及方言发展而来的特点和其相对自由的发展环境。然而,随着社会的发展,大众生活方式开始转变,再加上流行音乐文化的崛起,商品时尚化的渲染,给传统民歌带来了巨大的冲击;同时由于人们,尤其是青年人,日渐以普通话作为生活工作以及学习中的常用语言,民歌生长的土壤也逐渐流失\cite{yang2017浅析当代民歌艺术}。
		
		\indent 由此可见,尽管民歌有着悠长的发展历史与丰富多彩的文化积淀,在如今音乐市场趋于更加多元的情况下,仍然呈现出式微的态势,并离当代人们,尤其是青年人的生活越来越远。
	
		
		\section{如何让传统民歌更好地融入当代青年人的生活}
		\indent 通过第\ref{现状}节的分析不难发现发展民歌、让青年人多了解民歌的必要性与迫切性。针对以上这些情况,本提出以下几条建议。
		
			\subsection{教育上的引导}
			\indent 当代青年人之所以不了解中国传统民歌,除了第\ref{现状}节所说的一系列原因之外,很大程度上也是由于缺乏相关教育者的引导。从小学到高中,音乐课甚至音乐教育都是要让步于其它课程的。在这样的大环境下,受到正规的音乐教育尚且不易,更何况是略显小众的传统民歌呢?而大学教育中,虽然与相关的课程,但大多数时候是作为众多选修课当中的一门,仍然不能成规模地向青年人传授民歌知识,激发其对于传统民歌的兴趣。
			
			\indent 因此,要想真正实现从教育上对青年人的引导,就需要从小学开始,逐步加强在音乐教育,尤其是民歌教育上的投入。让大众从小开始就能对民歌文化有一个正确的认识,并逐渐培养其兴趣,让青年人能够主动地去了解民歌,学习民歌。
				
			\subsection{加强民歌文化的宣传}
			\indent 除了尚在学校学习的学生们之外,还有大量已毕业甚至没有机会进入学校学习的青年人分布在社会各处。他们大多已参加工作,无心再去学校里参与系统地学习,自然也难以通过教育上的引导来促进他们去了解民歌文化。
				
			\indent 对于这样的青年人群体,需要通过媒体的力量,大力宣传传统民歌文化,吸引大众注意,调动其兴趣。具体而言,可以在电视上举办相关的比赛,让众多民歌歌手能够在公众视野里同台竞技、一展风采,让观众们可以同时领略到不同地区、不同民族的音乐魅力;还可以在电台中创建民歌科普类的节目,让人们在忙碌的间隙中能够通过民歌来放松身心、丰富知识;亦或是在如今众多的文化类节目中,穿插民歌相关的环节,让青年人们能够以传统文化的视角去感受民歌、理解民歌。
			
			
			\subsection{民歌自身的创新与发展}
		    \indent 除了不断引导青年人去了解外,民歌自身也需要不断地创新,与时俱进。不能囿于经典而抹杀歌手的特性,也不能拘泥于传统而忽视市场的需求。应当鼓励民歌唱法、表演形式的多元化发展。同时也可促进民歌与流行音乐的结合,在提升传统民歌人气的同时,也能丰富流行音乐的内涵。近来已有不少民歌表演中出现流行歌手的身影及其所带来的流行元素,不论其结果如何,这都是一种很好的尝试。
		    
		    \indent 但在顺应时代潮流的同时,也应注意不能过于追求满足市场需求而削弱甚至抛弃中国传统文化的内核。要坚守民歌纯粹、自然的本质,不能被商业化的浪潮主导。让民歌朝着一个科学的方向发展,既保留其本色,又能不断绽放出新的光彩。
		
		
		\section{总结}
	    \indent 我国民歌在经历了漫长的历史积淀之后,尽管在当代遭遇了一系列挑战,尤其是难以在青年人群体中顺利融入,但它的生命力依旧旺盛,发展前景依旧广阔。只要能够采取正确科学的措施,逐步引导青年人去了解学习民歌文化,同时再加强民歌方面的创新与多元化发展,民歌必然会受到全国乃至全球人民的关注,使之在世界音乐文化的舞台上屹立不倒。
	
	
%\begin{thebibliography}{}
%\addtolength{\itemsep}{-1.5ex}
%\bibitem [1]{Sketch2photo} Sketch2Photo T.Cao
%\end{thebibliography}


\end{multicols}

 \quad \\

\begin{multicols}{1}
\bibliography{ref.bib}
\end{multicols}



\end{document}